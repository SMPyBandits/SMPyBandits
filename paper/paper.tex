\documentclass[a4paper,10pt,]{article}

  % http://www.jmlr.org/format/format.html
  %
  % See http://jmlr.org/papers/v18/ for example of JMLR papers from last year
  %
  % Any additional packages needed should be included after jmlr2e.
  % Note that jmlr2e.sty includes epsfig, amssymb, natbib and graphicx,
  % and defines many common macros, such as 'proof' and 'example'.
  % It also sets the bibliographystyle to plainnat; for more information on
  % natbib citation styles, see the natbib documentation, a copy of which
  % is archived at http://www.jmlr.org/format/natbib.pdf
\usepackage{jmlr2e}

% \usepackage[]{palatino}
% \usepackage{amssymb}
\usepackage{amsmath}

\usepackage{ifxetex,ifluatex}
\ifnum 0\ifxetex 1\fi\ifluatex 1\fi=0 % if pdftex
  \usepackage[utf8]{inputenc}
  \usepackage[T1]{fontenc}
\else % if luatex or xelatex
  \usepackage{unicode-math}
  \defaultfontfeatures{Ligatures=TeX,Scale=MatchLowercase}
\fi

% use upquote if available, for straight quotes in verbatim environments
\IfFileExists{upquote.sty}{\usepackage{upquote}}{}

% use microtype if available
\IfFileExists{microtype.sty}{%
\usepackage[]{microtype}
\UseMicrotypeSet[protrusion]{basicmath} % disable protrusion for tt fonts
}{}

\usepackage{xcolor}
\definecolor{dkgreen}{rgb}{0,0.4,0}           % Define a new color rgb(0, 102, 0)
\definecolor{gray}{rgb}{0.5,0.5,0.5}          % Define a new color rgb(127, 127, 127)
\definecolor{mauve}{rgb}{0.58,0,0.82}         % Define a new color rgb(147, 0, 209)
\definecolor{darkgreen}{rgb}{0.00,0.70,0.00}  % Define a new color rgb(0,178,0)

% \usepackage[scale=0.71]{geometry}

\usepackage{lastpage,fancyhdr}     % customize the headers and footers
\pagestyle{fancy}
    \renewcommand{\headrulewidth}{0.2pt}
    \renewcommand{\footrulewidth}{0.2pt}
    \lhead{\emph{SMPyBandits} presentation paper}
    \rhead{\emph{\today}}
    \lfoot{\href{https://GitHub.com/SMPyBandits/SMPyBandits}{GitHub.com/SMPyBandits/SMPyBandits}}
    \cfoot{\(\thepage/\pageref{LastPage}\)}
    \rfoot{Lilian Besson}

% \usepackage{graphicx}
\graphicspath{{../}}
\DeclareGraphicsExtensions{.jpg,.pdf,.mps,.eps,.png}


\IfFileExists{parskip.sty}{%
\usepackage{parskip}
}{% else
\setlength{\parindent}{0pt}
\setlength{\parskip}{6pt plus 2pt minus 1pt}
}
\setlength{\emergencystretch}{3em}  % prevent overfull lines
\providecommand{\tightlist}{%
  \setlength{\itemsep}{0pt}\setlength{\parskip}{0pt}}
\setcounter{secnumdepth}{5}
% Redefines (sub)paragraphs to behave more like sections
\ifx\paragraph\undefined\else
\let\oldparagraph\paragraph
\renewcommand{\paragraph}[1]{\oldparagraph{#1}\mbox{}}
\fi
\ifx\subparagraph\undefined\else
\let\oldsubparagraph\subparagraph
\renewcommand{\subparagraph}[1]{\oldsubparagraph{#1}\mbox{}}
\fi

\usepackage{fancyvrb}
\newcommand{\VerbBar}{|}
\newcommand{\VERB}{\Verb[commandchars=\\\{\}]}
\DefineVerbatimEnvironment{Highlighting}{Verbatim}{commandchars=\\\{\}}
% Add ',fontsize=\small' for more characters per line
\newenvironment{Shaded}{}{}
\newcommand{\AlertTok}[1]{\textcolor[rgb]{1.00,0.00,0.00}{\textbf{#1}}}
\newcommand{\AnnotationTok}[1]{\textcolor[rgb]{0.38,0.63,0.69}{\textbf{\textit{#1}}}}
\newcommand{\AttributeTok}[1]{\textcolor[rgb]{0.49,0.56,0.16}{#1}}
\newcommand{\BaseNTok}[1]{\textcolor[rgb]{0.25,0.63,0.44}{#1}}
\newcommand{\BuiltInTok}[1]{#1}
\newcommand{\CharTok}[1]{\textcolor[rgb]{0.25,0.44,0.63}{#1}}
\newcommand{\CommentTok}[1]{\textcolor[rgb]{0.38,0.63,0.69}{\textit{#1}}}
\newcommand{\CommentVarTok}[1]{\textcolor[rgb]{0.38,0.63,0.69}{\textbf{\textit{#1}}}}
\newcommand{\ConstantTok}[1]{\textcolor[rgb]{0.53,0.00,0.00}{#1}}
\newcommand{\ControlFlowTok}[1]{\textcolor[rgb]{0.00,0.44,0.13}{\textbf{#1}}}
\newcommand{\DataTypeTok}[1]{\textcolor[rgb]{0.56,0.13,0.00}{#1}}
\newcommand{\DecValTok}[1]{\textcolor[rgb]{0.25,0.63,0.44}{#1}}
\newcommand{\DocumentationTok}[1]{\textcolor[rgb]{0.73,0.13,0.13}{\textit{#1}}}
\newcommand{\ErrorTok}[1]{\textcolor[rgb]{1.00,0.00,0.00}{\textbf{#1}}}
\newcommand{\ExtensionTok}[1]{#1}
\newcommand{\FloatTok}[1]{\textcolor[rgb]{0.25,0.63,0.44}{#1}}
\newcommand{\FunctionTok}[1]{\textcolor[rgb]{0.02,0.16,0.49}{#1}}
\newcommand{\ImportTok}[1]{#1}
\newcommand{\InformationTok}[1]{\textcolor[rgb]{0.38,0.63,0.69}{\textbf{\textit{#1}}}}
\newcommand{\KeywordTok}[1]{\textcolor[rgb]{0.00,0.44,0.13}{\textbf{#1}}}
\newcommand{\NormalTok}[1]{#1}
\newcommand{\OperatorTok}[1]{\textcolor[rgb]{0.40,0.40,0.40}{#1}}
\newcommand{\OtherTok}[1]{\textcolor[rgb]{0.00,0.44,0.13}{#1}}
\newcommand{\PreprocessorTok}[1]{\textcolor[rgb]{0.74,0.48,0.00}{#1}}
\newcommand{\RegionMarkerTok}[1]{#1}
\newcommand{\SpecialCharTok}[1]{\textcolor[rgb]{0.25,0.44,0.63}{#1}}
\newcommand{\SpecialStringTok}[1]{\textcolor[rgb]{0.73,0.40,0.53}{#1}}
\newcommand{\StringTok}[1]{\textcolor[rgb]{0.25,0.44,0.63}{#1}}
\newcommand{\VariableTok}[1]{\textcolor[rgb]{0.10,0.09,0.49}{#1}}
\newcommand{\VerbatimStringTok}[1]{\textcolor[rgb]{0.25,0.44,0.63}{#1}}
\newcommand{\WarningTok}[1]{\textcolor[rgb]{0.38,0.63,0.69}{\textbf{\textit{#1}}}}

\usepackage{marvosym}

% set default figure placement to htbp
\makeatletter
\def\fps@figure{htbp}
\makeatother


% -----------------------------------------------------------------
% \ShortHeadings{short title}{short authors}
\ShortHeadings{\emph{SMPyBandits} presentation paper}{Lilian Besson}

% Heading arguments are {volume}{year}{pages}{submitted}{published}{author-full-names}
% \jmlrheading{vol}{year}{pages}{Submitted date}{published date}{paper id}{authors}
% \jmlrheading{18}{2018}{1-5}{4/18}{??/18}{Lilian Besson} % dates/pages from editor
\jmlrshortheading{2018}{Lilian Besson}

% \firstpageno{1} % the pagenumber you are assigned to start with by the editor.


% -----------------------------------------------------------------
\begin{document}

\title{\emph{SMPyBandits}: a Research Framework for Single and Multi-Players Multi-Arms Bandits Algorithms in Python}

% -----------------------------------------------------------------
% \author{
% Lilian Besson\thanks{
% \textcolor{blue}{\Letter: \texttt{Lilian.Besson{[}AT}CentraleSupelec{[}.}fr}},
% \textcolor{darkgreen}{ORCID: \href{https://orcid.org/0000-0003-2767-2563}{\texttt{0000-0003-2767-2563}}}
% \newline
% PhD Student at CentraleSupélec, campus of Rennes, SCEE team \& Inria
% Lille Nord Europe, SequeL team.
% }
% }

\author{\name Lilian Besson \email {Lilian}{.}{Besson}{@}{CentraleSupelec}{.}{fr} \\
        \addr CentraleSup\'elec (campus of Rennes), IETR, SCEE Team,\\
        Avenue de la Boulaie -- CS $47601$, F-$35576$ Cesson-S\'evign\'e, France
}

% -----------------------------------------------------------------
% \date{23 March 2018}
% \date{}  % remove date

% -----------------------------------------------------------------
\editor{?}

\maketitle

\vspace*{15pt}

% -----------------------------------------------------------------
\begin{abstract}%   <- trailing '%' for backward compatibility of .sty file
  \emph{SMPyBandits} is a package for numerical simulations on
  \emph{single}-player and \emph{multi}-players
  \href{https://en.wikipedia.org/wiki/Multi-armed_bandit}{Multi-Armed
  Bandits (MAB)} algorithms \citep{Bubeck12}, written in
  \href{https://www.python.org/}{Python (2 or 3)} \citep{python}.
  \emph{SMPyBandits} is the most complete open-source implementation of
  state-of-the-art algorithms tackling various kinds of sequential
  learning problems referred to as Multi-Armed Bandits. It aims at being
  extensive, simple to use and maintain, with a clean and perfectly
  documented codebase. It allows fast prototyping of simulations and
  experiments, with an easy configuration system and command-line options
  to customize experiments while starting them (see below for an example).
\end{abstract}


% \begin{center}\rule{0.5\linewidth}{\linethickness}\end{center}

\section{Presentation}\label{presentation}

\subsection{Single-Player MAB}\label{single-player-mab}

Multi-Armed Bandit (MAB) problems are well-studied sequential decision
making problems in which an agent repeatedly chooses an action (the
``\emph{arm}'' of a one-armed bandit) in order to maximize some total
reward \citep{Robbins52}, \citep{LaiRobbins85}. Initial motivation for
their study came from the modeling of clinical trials, as early as 1933
with the seminal work of Thompson \citep{Thompson33}, where arms
correspond to different treatments with unknown, random effect. Since
then, MAB models have been proved useful for many more applications,
that range from cognitive radio \citep{Jouini09} to online content
optimization (news article recommendation \citep{Li10}, online
advertising \citep{LiChapelle11} or A/B Testing \citep{Kaufmann14},
\citep{Jamieson17}), or portfolio optimization \citep{Sani12}.

\emph{SMPyBandits} is the most complete open-source implementation of
single-player (classical) bandit algorithms
(\href{https://smpybandits.github.io/docs/Policies.html}{over 65!}). We
use a well-designed hierarchical structure and
\href{https://smpybandits.github.io/uml_diagrams/README.html}{class
inheritance scheme} to minimize redundancy in the codebase. Most
existing algorithms are index-based, and can be written very shortly by
inheriting from the
\href{https://smpybandits.github.io/docs/Policies.IndexPolicy.html}{\texttt{IndexPolicy}}
class.

\subsection{Multi-Players MAB}\label{multi-players-mab}

For Cognitive Radio applications, a well-studied extension is to
consider \(M\geq2\) players, interacting on the \emph{same} \(K\) arms.
Whenever two or more players select the same arm at the same time, they
all suffer from a collision. Different collision models has been
proposed, and the simplest one consist in giving a \(0\) reward to each
colliding players. Without any centralized supervision or coordination
between players, they must learn to access the \(M\) best resources
(\emph{i.e.}, arms with highest means) without collisions.

\emph{SMPyBandits} implements
\href{https://smpybandits.github.io/docs/Environment.CollisionModels.html}{all
the collision models} found in the literature, as well as all the
algorithms from the last 10 years or so (including
\href{https://smpybandits.github.io/docs/PoliciesMultiPlayers.rhoRand.html}{\texttt{rhoRand}}
from 2009,
\href{https://smpybandits.github.io/docs/Policies.MEGA.html}{\texttt{MEGA}}
from 2015,
\href{https://smpybandits.github.io/docs/Policies.MusicalChair.html}{\texttt{MusicalChair}}
from 2016, and our state-of-the-art algorithms
\href{https://smpybandits.github.io/docs/PoliciesMultiPlayers.RandTopM.html}{\texttt{RandTopM}}
and
\href{https://smpybandits.github.io/docs/PoliciesMultiPlayers.MCTopM.html}{\texttt{MCTopM}})
from \citet{BessonALT2018}.

% \begin{center}\rule{0.5\linewidth}{\linethickness}\end{center}

\section{Features}\label{features}

With this numerical framework, simulations can run on a single CPU or a
multi-core machine using joblib \citep{joblib}, and summary plots are
automatically saved as high-quality PNG, PDF and EPS (ready for being
used in research article), using matplotlib \citep{matplotlib} and
seaborn \citep{seaborn}. Making new simulations is very easy, one only
needs to write a configuration script and no knowledge of the internal
code architecture.

\subsection{Documentation}\label{documentation}

A complete sphinx \citep{sphinx} documentation for each algorithms and
every piece of code, included the constants in the different
configuration files, is available here:
\href{https://smpybandits.github.io/}{\texttt{https://SMPyBandits.GitHub.io}}.

\subsection{How to run the experiments
?}\label{how-to-run-the-experiments}

For example, this short bash snippet\footnote{~See
  \href{https://smpybandits.github.io/How_to_run_the_code.html}{this
  page of the documentation} for more details.} shows how to clone the
code\footnote{~SMPyBandits is also available on Pypi, see
  \href{https://pypi.org/project/SMPyBandits/}{pypi.org/project/SMPyBandits}.
  You can install it directly with
  \texttt{sudo\ pip\ install\ SMPyBandits}.}, install the requirements
for Python 3 (in a virtualenv \citep{virtualenv}), and starts some
simulation for \(N=1000\) repetitions of the default non-Bayesian
Bernoulli-distributed problem, for \(K=9\) arms, an horizon of
\(T=10000\) and on \(4\) CPUs\footnote{~It takes about \(20\) to
  \(40\) minutes for each simulation, on a standard \(4\)-cores \(64\)
  bits GNU/Linux laptop.}. Using environment variables (\texttt{N=1000})
when launching the simulation is not required but it is convenient.

\begin{Shaded}
\begin{Highlighting}[]
\CommentTok{# 1. get the code in /tmp/, or wherever you want}
\BuiltInTok{cd}\NormalTok{ /tmp/}
\FunctionTok{git}\NormalTok{ clone https://GitHub.com/SMPyBandits/SMPyBandits.git}
\BuiltInTok{cd}\NormalTok{ SMPyBandits.git}
\CommentTok{# 2. just be sure you have the latest virtualenv from Python 3}
\FunctionTok{sudo}\NormalTok{ pip3 install --upgrade virtualenv}
\CommentTok{# 3. create and active the virtualenv}
\ExtensionTok{virtualenv}\NormalTok{ venv}
\BuiltInTok{.} \ExtensionTok{venv/bin/activate}
\CommentTok{# 4. install the requirements in the virtualenv}
\ExtensionTok{pip}\NormalTok{ install -r requirements.txt}
\CommentTok{# Alternative to 1 and 4: pip install SMPyBandits in the virtualenv}
\CommentTok{# 5. run a single-player simulation!}
\VariableTok{N=}\NormalTok{1000 }\VariableTok{T=}\NormalTok{10000 }\VariableTok{K=}\NormalTok{9 }\VariableTok{N_JOBS=}\NormalTok{4 }\FunctionTok{make}\NormalTok{ single}
\end{Highlighting}
\end{Shaded}

\subsection{Example of simulation and
illustration}\label{example-of-simulation-and-illustration}

A small script
\href{https://smpybandits.github.io/docs/configuration.html}{\texttt{configuration.py}}
is used to import the
\href{https://smpybandits.github.io/docs/Arms.html}{arm classes}, the
\href{https://smpybandits.github.io/docs/Policies.html}{policy classes}
and define the problems and the experiments. For instance, we can
compare the standard anytime
\href{https://smpybandits.github.io/docs/Policies.klUCB.html}{\texttt{klUCB}}
algorithm against the non-anytime variant
\href{https://smpybandits.github.io/docs/Policies.klUCBPlusPlus.html}{\texttt{klUCBPlusPlus}}
algorithm, as well as
\href{https://smpybandits.github.io/docs/Policies.UCBalpha.html}{\texttt{UCB}}
(with \(\alpha=1\)) and
\href{https://smpybandits.github.io/docs/Policies.Thompson.html}{\texttt{Thompson}}
(with
\href{https://smpybandits.github.io/docs/Policies.Posterior.Beta.html}{Beta
posterior}). Figure \ref{fig:plot1} below shows the average regret\footnote{~The regret is the difference between the cumulated
  rewards of the best fixed-armed strategy (which is the oracle strategy
  for stationary bandits) and the cumulated rewards of the considered
  algorithms.} for these \(4\) algorithms, and the asymptotic lower-bound from \citep{LaiRobbins85}.

\begin{figure}
\centering
\includegraphics[width=0.95\textwidth]{plots/paper/1.png}
\caption{\small Single-player simulation showing the regret of $4$ algorithms. They all perform very well, and at finite time they are empirically \emph{below} the asymptotic lower-bound. Each algorithm is known to be order-optimal (\emph{i.e.}, its regret is proved to match the lower-bound up-to a constant), and each but UCB is known to be optimal (\emph{i.e.} with the constant matching the lower-bound).}
\label{fig:plot1}
\end{figure}

% \begin{center}\rule{0.5\linewidth}{\linethickness}\end{center}

\section{\texorpdfstring{Research using \emph{SMPyBandits}}{Research using SMPyBandits}}\label{research-using-smpybandits}

\emph{SMPyBandits} was used for the following research articles since
\(2017\):

\begin{itemize}
\tightlist
\item
  For \citet{BessonALT2018}, we used \emph{SMPyBandits} for all the
  simulations for multi-player bandit algorithms\footnote{~See
    the page
    \href{https://smpybandits.github.io/MultiPlayers.html}{\texttt{MultiPlayers}}
    on the documentation.}. We designed the two
  \href{https://smpybandits.github.io/docs/PoliciesMultiPlayers.RandTopM.html}{\texttt{RandTopM}}
  and
  \href{https://smpybandits.github.io/docs/PoliciesMultiPlayers.MCTopM.html}{\texttt{MCTopM}}
  algorithms and proved than they enjoy logarithmic regret in the usual
  setting, and outperform significantly the previous state-of-the-art
  solutions (\emph{i.e.},
  \href{https://smpybandits.github.io/docs/PoliciesMultiPlayers.rhoRand.html}{\texttt{rhoRand}},
  \href{https://smpybandits.github.io/docs/Policies.MEGA.html}{\texttt{MEGA}}
  and
  \href{https://smpybandits.github.io/docs/Policies.MusicalChair.html}{\texttt{MusicalChair}}).
\end{itemize}

\begin{itemize}
\tightlist
\item
  In \citet{BessonWCNC2018}, we used \emph{SMPyBandits} to illustrate and
  compare different aggregation algorithms\footnote{~See the page
    \href{https://smpybandits.github.io/Aggregation.html}{\texttt{Aggregation}}
    on the documentation.}. We designed a variant of the Exp3 algorithm
  for online aggregation of experts \citep{Bubeck12}, called
  \href{https://smpybandits.github.io/docs/Policies.Aggregator.html}{\texttt{Aggregator}}.
  Aggregating experts is a well-studied idea in sequential learning and
  in machine learning in general. We showed that it can be used in
  practice to select on the run the best bandit algorithm for a certain
  problem from a fixed pool of experts. This idea and algorithm can have
  interesting impact for Opportunistic Spectrum Access applications
  \citep{Jouini09} that use multi-armed bandits algorithms for sequential
  learning and network efficiency optimization.
\end{itemize}

\begin{itemize}
\tightlist
\item
  In \citet{Besson2018c}, we used \emph{SMPyBandits} to illustrate and
  compare different ``doubling trick'' schemes\footnote{~See the
    page
    \href{https://smpybandits.github.io/DoublingTrick.html}{\texttt{DoublingTrick}}
    on the documentation.}. In sequential learning, an algorithm is
  \emph{anytime} if it does not need to know the horizon \(T\) of the
  experiments. A well-known trick for transforming any non-anytime
  algorithm to an anytime variant is the ``Doubling Trick'': start with
  an horizon \(T_0\in\mathbb{N}^*\), and when \(t > T_i\), use
  \(T_{i+1} = 2 T_i\). We studied two generic sequences of growing
  horizons (geometric and exponential), and we proved two theorems that
  generalized previous results. A geometric sequence suffices to minimax
  regret bounds (in \(R_T = \mathcal{O}(\sqrt{T})\)), with a constant
  multiplicative loss \(\ell \leq 4\), but cannot be used to conserve a
  logarithmic regret bound (in \(R_T = \mathcal{O}(\log(T))\)). And an
  exponential sequence can be used to conserve logarithmic bounds, with
  a constant multiplicative loss also \(\ell \leq 4\) in the usual
  setting. It is still an open question to know if a well-tuned
  exponential sequence can conserve minimax bounds or weak minimax
  bounds (in \(R_T = \mathcal{O}(\sqrt{T \log(T)})\)).
\end{itemize}

\section{Dependencies}\label{dependencies}

Written in Python \citep{python} (versions \emph{2.7+} or \emph{3.4+}),
using \texttt{matplotlib} \citep{matplotlib} for 2D plotting,
\texttt{numpy} \citep{numpy} for data storing, random number generations
and and operations on arrays, \texttt{scipy} \citep{scipy} for
statistical and special functions, and \texttt{seaborn} \citep{seaborn}
for pretty plotting and colorblind-aware colormaps. Optional
dependencies include \texttt{joblib} \citep{joblib} for parallel
simulations, \texttt{numba} \citep{numba} for automatic speed-up on small
functions, \texttt{sphinx} \citep{sphinx} for generating the
documentations, \texttt{virtualenv} \citep{virtualenv} for launching
simulations in isolated environments, and \texttt{jupyter}
\citep{jupyter} used with \texttt{ipython} \citep{ipython} to experiment
with the code.

\bibliography{paper.bib}

\end{document}
